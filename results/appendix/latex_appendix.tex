\documentclass[10pt,landscape]{article}
\usepackage[utf8]{inputenc}
\usepackage{longtable}[=v4.13]%
\usepackage{tabu,booktabs}
\usepackage{siunitx}
\usepackage[top=1in,bottom=1in,left=1in,right=1in]{geometry}
\usepackage{multirow}
\usepackage{dcolumn}
\usepackage{subfig, float}
\usepackage[english]{babel}
\usepackage[autostyle, english = american]{csquotes}
\usepackage{caption}
\usepackage{rotating}
\usepackage{soul,color}
\usepackage[nottoc,numbib]{tocbibind}
\usepackage{pdflscape} 
\usepackage{caption}
\usepackage{color,soul}
\usepackage{fancyhdr}
\usepackage{graphicx}
\usepackage[authordate,backend=bibtex,isbn=false]{biblatex-chicago}
\addbibresource{bib.bib}

\newcolumntype{d}{S[
    input-open-uncertainty=,
    input-close-uncertainty=,
    parse-numbers = false,
    table-align-text-pre=false,
    table-align-text-post=false
 ]}

\begin{document}
%\begin{landscape}
\tableofcontents
\listoffigures
\listoftables
\clearpage

\section{Originating regressions for Figure 1}
\input{../tables/tex/fig1_table.tex}
\clearpage

\section{Originating regressions for Figure 2}
\input{../tables/tex/fig2_table.tex}
\clearpage

\section{Originating regressions for Figure 3}
\input{../tables/tex/fig3_full_sample.tex}
\clearpage
\input{../tables/tex/fig3_dem_sample.tex}
\clearpage
\input{../tables/tex/fig3_ind_sample.tex}
\clearpage
\input{../tables/tex/fig3_rep_sample.tex}
\clearpage

\section{Originating regressions for Figure 4}
\input{../tables/tex/fig4_full_sample.tex}
\clearpage
\input{../tables/tex/fig4_dem_sample.tex}
\clearpage
\input{../tables/tex/fig4_ind_sample.tex}
\clearpage
\input{../tables/tex/fig4_rep_sample.tex}
\clearpage

\section{Changes in Sorting Over Time}
The analyses presented in the main text indicate a connection between party-ideology sorting and affective polarization and participation; at the same time, we find no evidence of a link between sorting and attitudes about democracy.
However, the data we have used are all observational and cross-sectional. 
As such, the strength of our claims depends on the assumption that there are no major confounders we have omitted from our analyses. 
Despite the extensive set of controls we have included, there may be other variables (measured or unmeasured) that we have not considered that would change our results.

One way to address this issue is to consider changes in sorting in the same individual over time and connect those changes to these attitudes about democracy. 
This requires repeated observation of the same individual at more than one point in time and a consistent measure of sorting at both points.
Fortunately, one component of the 2020 ANES was a re-interview of a group of more than 2,000 respondents to the 2016 ANES.
This panel of respondents provides us with the opportunity to evaluate the effects of sorting to confirm our cross-sectional analyses.

Using these data, we continue to observe no link between changes in partisan sorting and attitudes about democracy. 
We take this as confirmatory evidence of the analyses presented in the main text.

\input{../tables/tex/a14.tex}


\clearpage
\section{Histograms of partisan-ideological sorting}

\begin{figure}[h]
    \centering
    \includegraphics[width=0.6\textwidth]{../figures/sorting_distributions.pdf}
    \caption{Distribution of partisan-ideological sorting, 2012--2020.}
    \label{fig:sorting_distribution}
\end{figure}

\clearpage
\section{Alternate model specifications}

Some of Mason’s models also include partisan ID strength and symbolic ideology. 
In the main results presented in text, we considered an alternate specification, similar to some of the models in \textcite{mason_uncivil_2018}.
In this alternate specification, partisan ID strength would be included in the full-sample models and (symbolic) ideology would be included in the partisan-subsample models.
(Symbolic ideology is included in the full-sample models shown below.)
As partisan ID strength and ideology strength are components of the sorting measure and therefore likely to dominate regressions, we elected not to include these. 

The models using the alternate specification are shown below.
In general, the results are quite similar, with the primary difference being their greatly increased \(R^2\)s.
For example, in the originating regression for the full-sample panel of Figure 1, the \(R^2\) in our preferred specification is 0.312, against 0.479 in the alternate specification.
This difference is more pronounced in the partisan subsamples (0.236 vs 0.763 for 2016 Republicans), because symbolic ideology is a strong predictor in partisan subsamples.

\clearpage
\begin{figure}[h]
    \centering
    \includegraphics[width=0.8\textwidth]{../figures/sorting_predictors_2016_supp.pdf}
    \caption{Alternate model specification for Figure 1.}
    \label{fig:sorting_predictors_2016_supp}
\end{figure}

\begin{figure}[h]
    \centering
    \includegraphics[width=0.8\textwidth]{../figures/sorting_predictors_2020_supp.pdf}
    \caption{Alternate model specification for Figure 2.}
    \label{fig:sorting_predictors_2020_supp}
\end{figure}

\begin{figure}[h]
    \centering
    \includegraphics[width=0.8\textwidth]{../figures/sorting_outcomes_2016_supp.pdf}
    \caption{Alternate model specification for Figure 3.}
    \label{fig:sorting_outcomes_2016_supp}
\end{figure}

\begin{figure}[h]
    \centering
    \includegraphics[width=0.8\textwidth]{../figures/sorting_outcomes_2020_supp.pdf}
    \caption{Alternate model specification for Figure 4.}
    \label{fig:sorting_outcomes_2020_supp}
\end{figure}

\clearpage
\input{../tables/tex/fig1_supp_table.tex}
\clearpage

\input{../tables/tex/fig2_supp_table.tex}
\clearpage

\input{../tables/tex/fig3_full_sample_supp.tex}
\clearpage
\input{../tables/tex/fig3_dem_sample_supp.tex}
\clearpage
\input{../tables/tex/fig3_ind_sample_supp.tex}
\clearpage
\input{../tables/tex/fig3_rep_sample_supp.tex}
\clearpage

\input{../tables/tex/fig4_full_sample_supp.tex}
\clearpage
\input{../tables/tex/fig4_dem_sample_supp.tex}
\clearpage
\input{../tables/tex/fig4_ind_sample_supp.tex}
\clearpage
\input{../tables/tex/fig4_rep_sample_supp.tex}
\clearpage

\section{Details on LASSO models}

We use the \texttt{glmnet} package in R to estimate all models \parencite{simon_regularization_2011}. 
The LASSO model is parameterized by a penalty term \(\lambda\) that tunes the amount of regularization in the model; LASSO regularizes by removing (i.e., setting the associate coefficient to zero) terms that do not improve predictive performance.
To determine the appropriate value of \(\lambda\), we use 10-fold cross-validation, sweeping across a range of candidate \(\lambda\)s and selecting the value that minimizes the error when averaging over the cross-validation folds.
In the tables below, we report the LASSO coefficients for the model with this \(\lambda\), as well as the McFadden pseudo-\(R^2\) for comparison with the OLS models.
Terms that the LASSO model has regularized out are represented with a `---'. 
(If a term in the LASSO model is zero, that means that it was not penalized out---it has some predictive accuracy when included---but its coeffcient is zero or very close to zero.)

In general, the OLS and LASSO models are quite similar; the McFadden pseudo-\(R^2\) of the LASSO models is often comparable to the adjusted \(R^2\) of the OLS models, and terms that are statistically significant in the OLS models tend to be the terms not excluded by the LASSO models. 
\clearpage


\section{Supplementary Tables}

\input{../tables/tex/a1.tex}
\clearpage

\input{../tables/tex/a2.tex}
\clearpage

\input{../tables/tex/a3.tex}
\clearpage

\input{../tables/tex/a4.tex}
\clearpage

\input{../tables/tex/a4_lasso.tex}
\clearpage

\input{../tables/tex/a4_16.tex}
\clearpage

\input{../tables/tex/a4_lasso_16.tex}
\clearpage

\input{../tables/tex/a5a.tex}
\clearpage

\input{../tables/tex/a5b.tex}
\clearpage

\input{../tables/tex/a6.tex}
\clearpage

\input{../tables/tex/a7.tex}
\clearpage

\input{../tables/tex/a8a.tex}
\clearpage

\input{../tables/tex/a8b.tex}
\clearpage

\input{../tables/tex/a9a.tex}
\clearpage

\input{../tables/tex/a9b.tex}
\clearpage

\input{../tables/tex/a10a.tex}
\clearpage

\input{../tables/tex/a10b.tex}
\clearpage

\input{../tables/tex/a11a.tex}
\clearpage

\input{../tables/tex/a11b.tex}
\clearpage

\input{../tables/tex/a12a.tex}
\clearpage

\input{../tables/tex/a12b.tex}
\clearpage

\input{../tables/tex/a13a.tex}
\clearpage

\input{../tables/tex/a13b.tex}
\clearpage

%\input{../tables/tex/mason_a1a.tex}
%\clearpage

%\input{../tables/tex/mason_a1b.tex}
%\clearpage

\input{../tables/tex/mason_a4.tex}
\clearpage

\input{../tables/tex/mason_replication.tex}
\clearpage
\input{../tables/tex/mason_replication_lasso.tex}
\clearpage

\input{../tables/tex/disagg_media_partisan_2016.tex}
\clearpage

\input{../tables/tex/disagg_media_partisan_2020.tex}
\clearpage

\input{../tables/tex/disagg_media.tex}
\clearpage

%\section{References}
\printbibliography

%\end{landscape}

\end{document}
